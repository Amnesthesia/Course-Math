\documentclass[8pt,a4paper]{article}
\usepackage[utf8x]{inputenc}
\usepackage{amsmath,amssymb,amsthm,enumitem,amsfonts,cases,array}
\author{Victor Rudolfsson}
\title{Matematikk Oblig}
\begin{document}
\begin{enumerate}[itemsep=20pt]
\item
	\begin{enumerate}
		\item 
				\text{Divide by 132:}\\
				$43252/132=327.6...7≃327$\\
				\text{Subtract by 132 times floored result:}\\
				$43252\%132=43252-132*327=88$ \hfill
		\item
				$-430/19=-22.6315$\\
				$(-430)\%19=(-430)-19*-23=7$\\
	\end{enumerate}
\item
	\begin{enumerate}
		\item
				$1999/7=285.571$\\
				$1999\%7=1999-7*285=4$\\
				\textbf{Answer: } Today is tuesday. Therefore, the answer is Saturday.\\
		\item
				$1493/24=62.2$\\
				$1493\%24=1493-24*62=5$\\
				\textbf{Answer: } Since the time is now 16:30, we add five hours and get 21:30
	\end{enumerate}
\item
	\begin{enumerate}
		\item
			\begin{tabular}[t]{l r}
			\((10495*43852*12442 + 42346*12993*165243)\%13\) \(=\) &\\
															 \(=\) \(((10495\%13)*(43852\%13)*(12442\%13))\%13\) \\
															 \(+\) \(((42346\%13)*(12993\%13)*(165243\%13))\%13\) \\
															 \(=\) \((4*3*1)\%13 + (5*6*0)\%13\)\\
															 \(=\) \((12+0)\%13\)\\
															 \(=\) \(12\)\\
			\end{tabular}
		\item
			\begin{tabular}[t]{ r c l}
			\( (10001^{100})\%13\) & \(=\) & \(10001\%13)^{100}\%13\) \\
								 & \(=\) & \(4^{100}\%13 \) \\
								 & \(=\) & \(4^10)^{10}\%13\) \\
								 & \(=\) & \(1048576^{10}\%13\)\\
								 & \(=\) & \(9\)\\
			\end{tabular}
	\end{enumerate}
\item
	\begin{enumerate}
		\item	
			\begin{tabular}[t]{l c r}
			\((1010101001)_2\)	& \(=\) & \((1*(2^0))\) \\
								& \(+\) & \((0*(2^1))\) \\
								& \(+\) & \((0*(2^2))\) \\
								& \(+\) & \((1*(2^3))\) \\
								& \(+\) & \((0*(2^4))\) \\
								& \(+\) & \((1*(2^5))\) \\
								& \(+\) & \((0*(2^6))\) \\
								& \(+\) & \((1*(2^7))\) \\
								& \(+\) & \((0*(2^8))\) \\
				 				& \(+\) & \((1*(2^9))\)\\
				 				& \(=\) & \( 681\)\\
			\end{tabular}
		\item
			\begin{tabular}[t]{l c r}
				\((ABCDEF)_16\) & \(=\) & \((15*(16^0))\)\\
								& \(+\) & \((14*(16^1))\)\\
								& \(+\) & \((13*(16^2))\)\\
								& \(+\) & \((12*(16^3))\)\\
								& \(+\) & \((11*(16^4))\)\\
								& \(+\) & \((10*(16^5))\)\\
								& \(=\) & \(11259375\)\\
			\end{tabular}
	\end{enumerate}
\item
	\begin{enumerate}
		\item
			$(0001 0000 0001)_2$\\
		\item
			$(1010 1011 1100 1101 1110 1111)_2$\\
	\end{enumerate}
\item
	\begin{enumerate}
		\item
			\begin{tabular}[t]{l}
				\(\hspace{3 mm}1\)\\
				\(\hspace{3 mm}FF\)\\
				+\(FF\)\\
				\(\line(1,0){30}\)\\
				\(1FE\)\\
			\end{tabular}
		\item
			$(101011)_2 * 101_2 = $
			\begin{tabular}[t]{l}
				\(101011\)\\
				\($*$\hspace{3 mm}101\)\\
				\(\line(1,0){30}\)\\
				\(101011\)\\
				\(000000\)\\
				\(\hspace{-3.5mm}10101100\)\\
				\(\hspace{-3.5mm}\line(1,0){40}\)\\
				\(\hspace{-3.5mm}11010111\)\\
			\end{tabular}
	\end{enumerate}
\item

	Personally I prefer using the $N\%(Max-Min)+Min$ formula\\ ($[0-32767]\%(225-105)+105$), because it allows me to easily specify a minimum and a maximum value. This is 				 because the highest possible value from $N\%(225-105)$ will always be 119; because as soon as it hits 120 it will result in the lowest possible value (0). We add 					 $105$, meaning the lowest possible value is now 105, and the highest possible is now 224.		
\item
	\begin{tabular}[t]{l c l}
		\($$2^150\%11$$\) & \(=\) &\($$2^{2*75}\%11$$\)\\
					  & \(=\) &\($$4^{75}\%11$$\)\\
					  & \(=\) & \($$4^{2*37+1}\%11$$\)\\
					  & \(=\) & \($$16^{37}*4\%11$$\)\\
					  & \(=\) & \($$5^{37}*4\%11$$\)\\
					  & \(=\) & \($$5^{2*18+1}\%11$$\)\\
					  & \(=\) & \($$25^{18}*4*5\%11$$\)\\
					  & \(=\) & \($$25^{18}*20\%11$$\)\\
					  & \(=\) & \($$3^{18}*9\%11$$\)\\
					  & \(=\) & \($$3^{2*9}*9\%11$$\)\\
					  & \(=\) & \($$9^9*9\%11$$\)\\
					  & \(=\) & \($$9^{2*4+1}*9\%11$$\)\\
					  & \(=\) & \($$81^4*9*9\%11$$\)\\
					  & \(=\) & \($$81^4*11\%11$$\)\\
					  & \(=\) & \($$4^4*4\%11$$\)\\
					  & \(=\) & \($$4^{2*2}*4\%11$$\)\\
					  & \(=\) & \($$16^2*4\%11$$\)\\
					  & \(=\) & \($$5^2*4\%11$$\)\\
					  & \(=\) & \($$25*4\%11$$\)\\
					  & \(=\) & \($$3*4\%11$$\)\\
					  & \(=\) & \($$12\%11$$\)\\
					  & \(=\) & \($$1$$\)\\
	\end{tabular}
\item
	\begin{tabular}[t]{l c l c r}
		\($$2^k\%14$$\) & \($$=$$\) & \($$12$$\) & \(\) & \(\) \\
		\($$2^1\%14$$\) & \($$=$$\) & \($$2$$\) & \(\) & \(\) \\
		\($$2^2\%14$$\) & \($$=$$\) & \($$4$$\) & \(\) & \(\) \\
		\($$2^3\%14$$\) & \($$=$$\) & \($$8$$\) & \(\) & \(\) \\
		\($$2^4\%14$$\) & \($$=$$\) & \($$16\%14$$\) & \($$=$$\) & \($$2$$\)\\
		\($$2^5\%14$$\) & \($$=$$\) & \($$32\%14$$\) & \($$=$$\) & \($$4$$\)\\
		\($$2^6\%14$$\) & \($$=$$\) & \($$64\%14$$\) & \($$=$$\) & \($$8$$\)\\
		\($$2^7\%14$$\) & \($$=$$\) & \($$128\%14$$\) & \($$=$$\) & \($$2$$\)\\
		\($$2^8\%14$$\) & \($$=$$\) & \($$256\%14$$\) & \($$=$$\) & \($$4$$\)\\
	\end{tabular}\\
	\textbf{Answer: }\text{There is no solution to $2^k\%14=12$}\\
\item
	\begin{tabular}[t]{l c r}
		\($$2^k\%11$$\) & \($$=$$\) & \($$1$$\)\\
		\($$2^2\%11$$\) & \($$=$$\) & \($$4$$\)\\
		\($$2^3\%11$$\) & \($$=$$\) & \($$8$$\)\\
		\($$2^4\%11$$\) & \($$=$$\) & \($$5$$\)\\
		\($$2^5\%11$$\) & \($$=$$\) & \($$10$$\)\\
		\($$2^6\%11$$\) & \($$=$$\) & \($$9$$\)\\
		\($$2^7\%11$$\) & \($$=$$\) & \($$7$$\)\\
		\($$2^8\%11$$\) & \($$=$$\) & \($$3$$\)\\
		\($$2^9\%11$$\) & \($$=$$\) & \($$6$$\)\\
		\($$2^10\%11$$\) & \($$=$$\) & \($$1$$\)\\
	\end{tabular}\\
	\textbf{Answer: }\text{$k=10$}\\			  	
\item
	$\sqrt{32623}=180.618$\\
	\text{Trying to find an integer by dividing by all prime numbers up to 180}\\
	
	$\dfrac{32623}{17}=1919$\hfill\text{Division by 17 gives an integer result}\\
	
	$\dfrac{1919}{101}=19$\\
	
	\textbf{Answer:} $32623=17*19*101$\\
\item
	$\gcd(30,31)=1$\\
	$\gcd(30,19)=1$\\
	$\gcd(30,17)=1$\\
	$\gcd(30,29)=1$\\
	$\gcd(30,37)=1$\\
\item
	\begin{enumerate}
		\item
			$\gcd(10024,4900):$\\
			\begin{tabular}[t]{l c r}
				\($$10024$$\) & \(=\) & \($$4900*2+224$$\)\\
				\($$4900$$\) & \(=\) & \($$224*21+196$$\)\\
				\($$224$$\) & \(=\) & \($$196*1+28$$\)\\
				\($$196$$\) & \(=\) & \($$28*7+0$$\)\\
				\($$\gcd(10024,4900)$$\) & \(=\) & \($$28$$\)\\
			\end{tabular}
		\item
			$\gcd(738,387):$\\
			\begin{tabular}[t]{l c r}
				\($$738$$\) & \(=\) & \($$387*1+351$$\)\\
				\($$387$$\) & \(=\) & \($$351*1+36$$\)\\
				\($$351$$\) & \(=\) & \($$36*9+27$$\)\\
				\($$36$$\) 	& \(=\) & \($$27*1+9$$\)\\
				\($$27$$\) 	& \(=\) & \($$9*3+0$$\)\\ 				
				\($$\gcd(738,387)$$\) & \(=\) & \($$9$$\)\\
			\end{tabular}
		\item
			$\gcd(4386,47874):$\\
			\begin{tabular}[t]{l c r}
				\($$47874$$\) & \(=\) & \($$4386*10+4014$$\)\\
				\($$4386$$\) & \(=\) & \($$4014*1+372$$\)\\
				\($$4014$$\) & \(=\) & \($$372*10+294$$\)\\
				\($$372$$\) & \(=\) & \($$294*1+78$$\)\\
				\($$294$$\) 	& \(=\) & \($$78*3+60$$\)\\
				\($$78$$\) 	& \(=\) & \($$60*1+18$$\)\\ 				
				\($$60$$\) 	& \(=\) & \($$18*3+6$$\)\\
				\($$18$$\) 	& \(=\) & \($$6*3+0$$\)\\
				\($$\gcd(4386,47874)$$\) & \(=\) & \($$6$$\)\\
			\end{tabular}	
	\end{enumerate}
\item
	$\gcd(4386,47874) = 4386x+47874y$\\
	\text{Taking the results from 13.b and solving for the remainders of each: }\\
	$4014 = 47874+4386*(-10)$\\
	$372 = 4386+4014*(-1)$\\
	$294 = 4014+372*(-10)$\\
	$78 = 372+294*(-1)$\\
	$60 = 294+78*(-3)$\\
	$18 = 78+60*(-1)$\\
	$6 = 60*+18*(-3)$\\
	
	\text{Euclids extended algorithm gives us:}\\
	6 \begin{tabular}[t]{l l}
	  & \(=\) \($$60+18(-3) $$\)\\ 					
	  & \(=\) \( $$60+((78(1)+60(-1))*-3)$$\)\\ 		
	  & \(=\) \($$60+((78(-3)+60*(3))$$\)\\			
	  & \(=\) \($$60*4+78*(-3)$$\)\\ 					
	  & \(=\) \( $$(294+78*-3)*4+78*-3$$\)\\ 			
	  & \(=\) \($$294*4+78*-15$$\)\\
	  & \(=\) \($$294*4+(372+294*(-1))*-15$$\)\\ 		
	  & \(=\) \( $$294*4+(372*-15+294*15)$$\)\\ 		
	  & \(=\) \($$294*(19)+372*(-15)$$\)\\	  		
	  & \(=\) \($$(4014+372*(-10))*19+372*-15$$\)\\ 	
	  & \(=\) \( $$(4014*19)+372*-205$$\)\\ 			
	  & \(=\) \($$4014*19+((4386+4014*-1)*-205)$$\)\\
	  & \(=\) \($$4014*224+4386*-205$$\)\\ 			
	  & \(=\) \( $$(47874+4386*-10)*224+4386*-205$$\)\\
	  & \(=\) \($$47874*224+4386*-2445$$\)\\
	  \end{tabular}\\	  
	   \textbf{y} \text{=224}\\
	   \textbf{x} \text{=-2445}\\
	   
\item
	\begin{enumerate}
		\item
			\textbf{Answer: }\text{10 does not divide by 28 (gcd(10024,4900))}\\
		\item
			\textbf{Answer: }\text{1 does not divide by 28 (gcd(10024,4900))}\\
		\item
			$ \gcd(738,387):$\\
			\begin{tabular}[t]{l c r}
			\($$738=387*1+351$$\) 	& \(=>\) 	& \($$351=738+387*(-1)$$\)\\
			\($$387=351*1+36$$\) 	& \(=>\) 	&\($$36=387+351*(-1)$$\)\\
			\($$351=36*9+27$$\) 	& \(=>\) 	&\($$27=351+36*(-9)$$\)\\
			\($$36=27*1+9$$\) 		& \(=>\) 	& \($$9=36+27*(-1)$$\)\\
			\($$27=9*3+0$$\)\\
			\end{tabular}\\
			
			\text{Euclids extended algorithm gives us:}\\
			9
			\begin{tabular}[t]{l l l l l}
			  & \(=\) & \($$36+27*(-1)$$\) 		& \(=\) & \($$36+(351+36*(-9))*-1$$\)\\
			  & \(=\) & \($$36*10+351*-1$$\) 	& \(=\) & \($$(387+351*-1)*10+351*-1$$\)\\
			  & \(=\) & \($$387*10+351*-11$$\) 	& \(=\) & \($$387*10+(738+387*-1)*-11$$\)\\
			  & \(=\) & \($$387*(21)+738*(-11)$$\)\\
			\end{tabular}\\
			
			Because $\frac{27}{9}=3$, we get:\\
			$x=-11*3=-33$\\
			$y=21*3=63$\\
		\item
			\textbf{Answer: }\text{28 does not divide by 9}
	\end{enumerate}
		
\end{enumerate}
\end{document} 
   
